% arara : pdflatex
% arara : bibtex
% arara : pdflatex
% arara : pdflatex

\documentclass[a4paper]{article}
\usepackage[T1]{fontenc}
\usepackage[italian]{babel}

\usepackage{enumitem}   % Gestisce le liste numerata
\setlist[enumerate,1]{label=\arabic* )} 
\usepackage{amsmath}    % Usato pe la matematica
\usepackage{booktabs}   % Gestisce le tabelle
\usepackage{tikz-cd}    % Usato per creare grafiche
\usepackage{numprint}   % Usato per stampare i numeri con sepatori alle migliaia
\usepackage{listings}   % Usato per avere un enviroment per il codice

\definecolor{codegray}{rgb}{0.5,0.5,0.5}
\definecolor{codepurple}{rgb}{0.58,0,0.82}
\definecolor{backcolour}{rgb}{0.95,0.95,0.92}
\lstdefinestyle{mystyle}{
    backgroundcolor=\color{backcolour},   
    keywordstyle=\color{blue},
    numberstyle=\tiny\color{codegray},
    stringstyle=\color{codepurple},
    basicstyle=\ttfamily\footnotesize,
    numbers=left,    
    breaklines=true,                
    numbersep=-10pt,                  
    tabsize=2
}
\lstset{style=mystyle}

% \usepackage{showkeys}   % Usato per vedere le label
% \usepackage{showframe}  % Usato per vedere i frame

\newtheorem{theorem}{Teorema}[section]

\newcommand{\code}[1]{
    \texttt{#1}
}

\begin{document}

\title{Parallelizzazione dell'algoritmo di intersezione di Möller\-Trumbore}
\author{Gallina Roberto}
\date{06/11/2023}

\maketitle

\newpage

\tableofcontents

\newpage

\section{Algoritmo di Möller\-Trumbore}
L'algoritmo di intersezione di Möller\-Trumbore è un metodo usato per capire se un raggio (linea con un punto d'inzio e una direzione) intersecano un triangolo nello spazio 3D.
è usato nella Computer Graphics per determinare quali parti di modelli 3D sono visibili da un certo punto di vista.

\begin{figure}[htp]
    \begin{lstlisting}[language=c++]
    bool RayIntersectsTriangle(
        Vector3D rayOrigin, 
        Vector3D rayVector, 
        Triangle* inTriangle,
        Vector3D& outIntersectionPoint
    ){
        const float EPSILON = 0.0000001;
        Vector3D vertex0 = inTriangle->vertex0;
        Vector3D vertex1 = inTriangle->vertex1;  
        Vector3D vertex2 = inTriangle->vertex2;
        
        Vector3D edge1, edge2, h, s, q;
        edge1 = vertex1 - vertex0;
        edge2 = vertex2 - vertex0;
        
        float a, f, u, v;
        h = rayVector.crossProduct(edge2);
        a = edge1.dotProduct(h);
        
        if (a > -EPSILON && a < EPSILON)
            return false;
        
        f = 1.0 / a;
        s = rayOrigin - vertex0;
        u = f * s.dotProduct(h);
        
        if (u < 0.0 || u > 1.0)
            return false;
            
        q = s.crossProduct(edge1);
        v = f * rayVector.dotProduct(q);
            
        if (v < 0.0 || u + v > 1.0)
            return false;
        
        float t = f * edge2.dotProduct(q);
        
        if (t > EPSILON){
            // ray intersection
            outIntersectionPoint = rayOrigin + rayVector * t;
            return true;
        }else
            return false;
    }
    \end{lstlisting}
    \caption{Algoritmo di Möller\-Trumbore da Wikipedia}
\end{figure}

\newpage

\section{Prima semplice implementazione}
Prima di procedere con la prima implementazione, va do a creare le strutture e le funzioni basilari che serviranno nei vari calcoli.

\subsection{Struttura Point3D}
\begin{lstlisting}[language=c++]
    struct Point3D{
        double x;
        double y;
        double z;
    };
\end{lstlisting}

\subsection{Struttura Triangle}
\begin{lstlisting}[language=c++]
    struct Triangle {
        Point3D p1;
        Point3D p2;
        Point3D p3;
    };

\end{lstlisting}

\subsection{Differenza tra vettori}
\begin{lstlisting}[language=c++]
    Point3D difference(Point3D a, Point3D b){
        return {a.x - b.x, a.y - b.y, a.z - b.z};
    }
\end{lstlisting}

\subsection{Dot product}
\begin{lstlisting}[language=c++]
    double dotProduct(Point3D &v1, Point3D &v2) {
        return v1.x * v2.x + v1.y * v2.y + v1.z * v2.z;
    }
\end{lstlisting}

\subsection{Cross product}
\begin{lstlisting}[language=c++]
    Point3D crossProduct(Point3D &v1, Point3D &v2) {
        return {
            v1.y * v2.z - v1.z * v2.y, 
            v1.z * v2.x - v1.x * v2.z,
            v1.x * v2.y - v1.y * v2.x
        };
    }
\end{lstlisting}

\newpage

\subsection{Codice}
Vado quindi ad implementare la versione del codice vista precedentemente

\begin{lstlisting}[language=c++]
    bool rayIntersectsTriangle(
        Point3D rayOrigin, 
        Point3D rayVector,
        Triangle inTriangle
    ){
        const float EPSILON = 0.0000001;
        Point3D vertex0 = inTriangle.p1;
        Point3D vertex1 = inTriangle.p2;
        Point3D vertex2 = inTriangle.p3;

        Point3D edge1, edge2, h, s, q;
        double a, f, u, v;

        edge1 = difference(vertex1, vertex0);
        edge2 = difference(vertex2, vertex0);

        h = crossProduct(rayVector, edge2);
        a = dotProduct(edge1, h);

        if (a > -EPSILON && a < EPSILON)
            return false;

        f = 1.0 / a;
        s = difference(rayOrigin, vertex0);
        u = f * dotProduct(s, h);

        if (u < 0.0 || u > 1.0)
            return false;

        q = crossProduct(s, edge1);
        v = f * dotProduct(rayVector, q);
        if (v < 0.0 || u + v > 1.0)
            return false;

        double t = f * dotProduct(edge2, q);

        if (t > EPSILON)
            return true;

        return false;
    }
\end{lstlisting}

\newpage

Implemento ora il `main` verificando che l'algoritmo funzioni correttamente sui casi base

\subsubsection{Raggio interseca triangolo}

\begin{lstlisting}[language=c++]
    int main() {
        Point3D rayOrigin = {0.0, 0.0, 0.0};
        Point3D rayDirection = {0.0, 0.0, 2.0};
        
        Triangle triangle = {
            {-0.5, 0.5, 0.5}, 
            {0.5, 0.0, 0.5}, 
            {0.5, -1.0, 0.5}
        };
        
        cout << rayIntersectsTriangle(rayOrigin, rayDirection, triangle);

        return 0;
    }
\end{lstlisting}

\begin{lstlisting}[language=bash,backgroundcolor=\color{black},basicstyle=\ttfamily\footnotesize\color{green}]
    1
\end{lstlisting}

\subsubsection{Raggio non interseca triangolo}

\begin{lstlisting}[language=c++]
    ..
    Triangle triangle = {
        {-0.5, 0.5, 0.5},
        {0.5, 0.5, 0.5},
        {0.5, -0.2, 0.5}
    };
    .. 
\end{lstlisting}

\begin{lstlisting}[language=bash,backgroundcolor=\color{black},basicstyle=\ttfamily\footnotesize\color{green}]
    0
\end{lstlisting}

\begin{figure}[ht]
    \centering
    \includegraphics[width=0.48\textwidth]{images/intersect.png}
    \includegraphics[width=0.48\textwidth]{images/notintersect.png}
\end{figure}

\newpage

\section{Esempio complesso}

\subsection{Lettura dei file}

Vado ora a implementare delle funzioni per poter leggere i file dei punti e delle mesh, sappiamo infatti che il file \emph{verts.csv} è composto da tre numeri con virgola scritti in notazione scentifica.

\vspace{5pt}

\noindent\emph{Es. \footnotesize{-6.195182353258132935e-02,-5.577753782272338867e-01,5.792263150215148926e-01}}

\vspace{7pt}

Vado quindi a creare una funzione che dato il nome del file genera un vettore di punti.

\begin{lstlisting}[language=c++]
    vector<Point3D> readPoints(string filename) {
        vector<Point3D> punti;
        string linea;
        ifstream file(filename);
        if (file.is_open()) {
            while (getline(file, linea)) {
                Point3D punto;
                stringstream ss(linea);
                string valore;
                if (getline(ss, valore, ',')) {
                    punto.x = stod(valore);
                }
                if (getline(ss, valore, ',')) {
                    punto.y = stod(valore);
                }
                if (getline(ss, valore)) {
                punto.z = stod(valore);
                }
                punti.push_back(punto);
            }
            file.close();
        }
        return punti;
    }
\end{lstlisting}

\newpage

Allo stesso modo sappiamo che il file \emph{mesh.csv} è composto da tre numeri scritti in notazione scentifica, corrispondenti al numero (indice) del punto.

\vspace{5pt}

\noindent\emph{Es. \footnotesize{1.000000000000000000e+00,2.000000000000000000e+00,0.000000000000000000e+00}}

\vspace{7pt}

Vado quindi a creare una funzione che dato il nome del file e un vettore di punti genera un vettore di triangoli (mesh).

\begin{lstlisting}[language=c++]
    vector<Triangle> readTriangles(string filename, vector<Point3D> punti) {
        vector<Triangle> triangoli;
        string linea;
        ifstream file(filename);
        if (file.is_open()) {
            while (getline(file, linea)) {
                Triangle t;
                stringstream ss(linea);
                string valore;
                if (getline(ss, valore, ',')) {
                    index = (int) stod(valore);
                    t.p1 = punti[index];
                }
                if (getline(ss, valore, ',')) {
                    index = (int) stod(valore);
                    t.p2 = punti[index];
                }
                if (getline(ss, valore)) {
                    index = (int) stod(valore);
                    t.p3 = punti[index];
                }
                triangoli.push_back(t);
            }
            file.close();
        }
        return triangoli;
    }
\end{lstlisting}

Nel main uso queste funzioni per caricare i dati.

\begin{lstlisting}[language=c++]
    int main() {
        vector<Point3D> punti = readPoints("verts.csv");
        vector<Triangle> triangoli = readTriangles("meshes.csv", punti);
        .
    } 
\end{lstlisting}

\newpage

\subsection{Implementazione}

Vado ora a implementare una funzione che data l'origine, la direzzione e un vettore di triangoli verifica se quel raggio interseca almeno un triangolo

\begin{lstlisting}[language=c++]
    bool rayIntersectsAnyTriangle(Point3D origin, Point3D dir, vector<Triangle> triangles) {
        for (const Triangle &triangle : triangles) {
            bool r = rayIntersectsTriangle(origin, dir, triangle);
            if (r) {
                return true;
            }
        }
        return false;
    }
\end{lstlisting}

Nel main vado quindi a verificare se ogni punto interserca almeno un triangolo

\begin{lstlisting}[language=c++]
    int main() {
        vector<Point3D> punti = readPoints("verts.csv");
        vector<Triangle> triangoli = readTriangles("meshes.csv", punti);

        Point3D rayOrigin = {0.0, 0.0, 0.0};

        for (const Point3D &punto : punti) {
            if (punto.x > 0 || punto.x <= 0)
                cout << rayIntersectsAnyTriangle(rayOrigin, punto, triangoli) << endl;
        }
        
        return 0;
    } 
\end{lstlisting}

\subsection{Output su file}

Non rimane che aggiungere la stampa direttamente su un file.

\begin{lstlisting}[language=c++]
    int main(){
        ofstream oFile("out.txt");
        vector<Point3D> punti = readPoints("verts.csv");
        vector<Triangle> triangoli = readTriangles("meshes.csv", punti);
    
        Point3D rayOrigin = {0.0, 0.0, 0.0};
    
        for (const Point3D &punto : punti)
        {
            if (punto.x > 0 || punto.x <= 0)
                oFile << rayIntersectsAnyTriangle(rayOrigin, punto, triangoli) << endl;
        }
        oFile.close();
    
        return 0;
    }
\end{lstlisting}

\newpage

\section{Parallelizzazione}

\subsection{Introduzione a CUDA}
La prima cosa da parallelizzare è il ciclo che viene fatto su punti, ricordiamo infatti che per ogni punto viene verificato se il segmento che unisce l'origine con esso interseca qualche triangolo.

Vado quindi a creare un Kernel CUDA che presa l'origine, un punto e il vettore di triangoli (e una indirizzo dove dalvare il risultato) verifichi la condizione.

Ricordiamo che le parti principali per usare un Kernel Cuda sono:
\begin{itemize}
    \item Allocazione memoria del device
    \item Passaggio dati dall'host al device
    \item Passaggio dati dal device all'host
    \item Deallocazione memoria del device
\end{itemize}

\newpage

\subsubsection{Passaggio dati}

La prima versione del kernel servirà per verificare il correttto passaggio dei dati tra l'\emph{host} e il \emph{device} pertanto restituirà sempre \emph{true}.

Vado quindi a preparare i parametri da passare: il punto origine, un array di punti, un array di trinagoli, il numero di triangoli e un array di booleani che serve per salvare i valori di ritorno. Per fare ciò definisco alcuni puntatori per l'array di punti e di triangoli e l'array di ritorno.

\begin{lstlisting}[language=c++]
    Point3D *d_points;
    Triangle *d_triangles;
    bool *d_result;
\end{lstlisting}

Vado poi ad allocarne lo spazio.

\begin{lstlisting}[language=c++]
    cudaMalloc(&d_points, Np * sizeof(Point3D) );
    cudaMalloc(&d_triangles, Nt * sizeof(Triangle) );
    cudaMalloc(&d_result, Np * sizeof(bool) );
\end{lstlisting}

E copio i dati.

\begin{lstlisting}[language=c++]
    cudaMemcpy( 
        d_points, h_punti, 
        Np * sizeof(Point3D), 
        cudaMemcpyHostToDevice
    );
    cudaMemcpy( 
        d_triangles, 
        h_triangoli, 
        Nt * sizeof(Triangle), 
        cudaMemcpyHostToDevice
    );
\end{lstlisting}

Andrò poi ad eseguire il kernel (spiegato successivamente), prima però calcolo la dimensione dei blocchi

\begin{lstlisting}[language=c++]
    dim3 DimGrid(Np/256, 1, 1);
    if (Np % 256)
        DimGrid.x++;

    dim3 DimBlock(256, 1, 1);

    rayIntersectsAnyTrianglesKernel<<<DimGrid, DimBlock>>>(
        origin, 
        d_points, 
        d_triangles, 
        Nt, 
        d_result
    );
\end{lstlisting}

E coppiare i risultati.

\begin{lstlisting}[language=c++]
    cudaMemcpy(
        result, 
        d_result, 
        Np * sizeof(bool),
        cudaMemcpyDeviceToHost
    );
\end{lstlisting}

E infine libero lo spazio allocato.

\begin{lstlisting}[language=c++]
    cudaFree( d_points );
    cudaFree( d_triangles );
    cudaFree( d_result );
\end{lstlisting}


Il kernel sarà una funzione void e dovrà ricevere i parametri precedentemente preparato, ogni thread analizzerà un unico punto, per fare ciò calcola l'indice del punto che deve analizzare.

\begin{lstlisting}[language=c++]
    __global__
    void rayIntersectsAnyTrianglesKernel(
        Point3D rayOrigin, 
        Point3D *ps,
        Triangle *ts,
        int Nt,
        bool *result
    ){
        int idx = blockIdx.x * blockDim.x + threadIdx.x;

        result[idx] = true;
    }
\end{lstlisting}

\newpage

\subsection{Implementazione algoritmo nel Kernel}

Per definizione devo restituire \emph{true} se almeno un triangolo è intersecato dal raggio che unisce l'origine al punto, pertanto il kernel sarà un cilo su tutti i triangoli e se interseca restituirà true altrimenti a fine ciclo restituirà false.

\begin{lstlisting}[language=c++]
    __global__
    void rayIntersectsAnyTrianglesKernel(
        Point3D rayOrigin, 
        Point3D *ps,
        Triangle *ts,
        int Nt,
        bool *result
    ){
        int idx = blockIdx.x * blockDim.x + threadIdx.x;

        Point3D point = ps[idx];
        Point3D dir = point;

        result[idx] = false;
        
        for(int i = 0; i < Nt; i++){
            Triangle t = ts[i];
            
            bool r = rayIntersectsTriangle(rayOrigin, dir, t);
            if(r){
                result[idx] = true;
            }
        }	
    }
\end{lstlisting}

Implemento quindi la funzione che verifica l\'insersezione di un raggio in un triangolo, devo quindi definire le funzioni dotProduct, crossProduct e difference come funzioni \emph{\_\_device\_\_} in modo che i thread possono usarle.

\begin{lstlisting}[language=c++]
    __device__ 
    Point3D difference(Point3D a, Point3D b) {
        return {a.x - b.x, a.y - b.y, a.z - b.z};
    }

    __device__ 
    Point3D crossProduct(Point3D &v1, Point3D &v2){
        return {    v1.y * v2.z - v1.z * v2.y, 
            v1.z * v2.x - v1.x * v2.z,
            v1.x * v2.y - v1.y * v2.x};
    }

    __device__ 
    double dotProduct(Point3D &v1, Point3D &v2){
        return v1.x * v2.x + v1.y * v2.y + v1.z * v2.z;
    }
\end{lstlisting}

\newpage

Vado ora a creare una funzione partendo da quando visto sulla Wiki, usando le funzioni precedentemente create.

\begin{lstlisting}[language=c++]
    __device__ 
    bool rayIntersectsTriangle(
        Point3D rayOrigin, 
        Point3D rayVector,
        Triangle inTriangle
    ){
        const float EPSILON = 0.0000001;
        Point3D vertex0 = inTriangle.p1;
        Point3D vertex1 = inTriangle.p2;
        Point3D vertex2 = inTriangle.p3;

        Point3D edge1, edge2, h, s, q;
        double a, f, u, v;

        edge1 = difference(vertex1, vertex0);
        edge2 = difference(vertex2, vertex0);

        h = crossProduct(rayVector, edge2);
        a = dotProduct(edge1, h);

        if (a > -EPSILON && a < EPSILON)
            return false;

        f = 1.0 / a;
        s = difference(rayOrigin, vertex0);
        u = f * dotProduct(s, h);

        if (u < 0.0 || u > 1.0)
            return false;

        q = crossProduct(s, edge1);
        v = f * dotProduct(rayVector, q);
        if (v < 0.0 || u + v > 1.0)
            return false;

        double t = f * dotProduct(edge2, q);

        if (t > EPSILON)
            return true;
        return false;
    }
\end{lstlisting}

\newpage

\subsection{Uso della shared memory}
Attualmente ogni thread, per eseguire il calcolo, deve ciclare su tutti i trinagoli, che risiedono in memoria globale; questo comporta un notevole tempo d'accesso.
Vado quindi ad introdurre una memoria condivisa (shared) tra i vari thread, essa memorizzerrà alcuni triangoli in modo che ci siano meno accessi in memoria globale.

Aggiungo la dimensione della shared memory, e dentro la funzione eseguita dai vari thread aggiungo un array di triangoli condiviso.

\begin{lstlisting}[language=c++]
    #define SH_SIZE 512
\end{lstlisting}

\begin{lstlisting}[language=c++]
    __global__
    void rayIntersectsAnyTrianglesKernel(
        ..
    ){
        ..
        __shared__ Triangle sh_triangle[SH_SIZE];
        ..
    }
\end{lstlisting}

Essendo la dimensione della shared minore del numero totale di triangoli, sarà necessario caricare una parte di triangoli, attendere che tutti i thread finiscano di analizzarli, per poi caricare i successivi

\begin{lstlisting}[language=c++]
    __global__
    void rayIntersectsAnyTrianglesKernel(
        Point3D rayOrigin, 
        Point3D *ps,
        Triangle *ts,
        int Nt,
        bool *result
    ){
        int idx = blockIdx.x * blockDim.x + threadIdx.x;
    
        Point3D point = ps[idx];   
        Point3D dir = point;
        result[idx] = false;
        
        __shared__ Triangle sh_triangle[SH_SIZE];
        
        bool res = false;
        
        for(int m = 0; m < Nt/SH_SIZE; m++){
            if(idx < SH_SIZE){
                sh_triangle[idx] = ts[m * SH_SIZE + idx];
            }
            __syncthreads();
    
            for(int i = 0; i < SH_SIZE; i++){
                Triangle t = sh_triangle[i];
                
                bool r = rayIntersectsTriangle(rayOrigin, dir, t);
                if(r){
                    res = true;
                }
            }
            __syncthreads();
        }
        result[idx] = res;
    }
\end{lstlisting}

\newpage

\section{Metriche}

\subsection{Specifiche}

Tutte le esecuzioni sono state fatte su un pc con:
\begin{itemize}
    \item CPU: AMD Ryzen 7 5800x 8-Core (3.80GHz)
    \item GPU: NVIDIA GeForce RTX 3060 (circa 3500 CUDA Core da 1.70 GHz)
\end{itemize}

\subsection{Metriche}

\begin{figure}[h]
    \centering
    \begin{tabular}{@{} c c @{} }
        \toprule
        \textbf{Tipo di codice}     & \textbf{Tempo}                                                                                                                                 \\

        \midrule
        Sequenziale                 & \begin{tabular}{@{} c @{}}4391245824ns = 4391245\textmu s \\ 4391ms = 4.3s   \end{tabular} \\[10pt] \hline

        \midrule
        Parallelo senza SH          & \begin{tabular}{@{} c @{}}536918784ns = 536918\textmu s \\ 536ms = 0.5s   \end{tabular}    \\[10pt] \hline

        \midrule
        Parallelo con SH (Size 16)  & \begin{tabular}{@{} c @{}}362736128ns = 362736\textmu s \\ 362ms = 0.3s   \end{tabular}    \\[10pt] \hline

        Parallelo con SH (Size 32)  & \begin{tabular}{@{} c @{}}412990208ns = 412990\textmu s \\ 412ms = 0.4s   \end{tabular}    \\[10pt] \hline

        Parallelo con SH (Size 64)  & \begin{tabular}{@{} c @{}}392447744ns = 392447\textmu s \\ 392ms = 0.4s   \end{tabular}   \\[10pt] \hline

        Parallelo con SH (Size 128) & \begin{tabular}{@{} c @{}}411473152ns = 411473\textmu s \\ 411ms = 0.4s   \end{tabular}  \\[10pt] \hline

        Parallelo con SH (Size 256) & \begin{tabular}{@{} c @{}}299524300ns = 299524\textmu s \\ 299ms = 0.3s   \end{tabular}  \\[10pt] \hline
    \end{tabular}
\end{figure}

\subsection{Conclusioni}

Anche se siamo su un caso di studio abbastanza piccolo, l'esecuzione tramite CPU risulta essere circa 10 volte più lenta rispetto all'uso della GPU; inoltre implmentando la shared memory si vedono dei leggeri benefici, credo queste differenze verrebbero amplificate in un caso di studio più grande.

\end{document}